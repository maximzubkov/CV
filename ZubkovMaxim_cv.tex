%%%%%%%%%%%%%%%%%%%%%%%%%%%%%%%%%%%%%%%%%
% Medium Length Professional CV
% LaTeX Template
% Version 2.0 (8/5/13)
%
% This template has been downloaded from:
% http://www.LaTeXTemplates.com
%
% Original author:
% Trey Hunner (http://www.treyhunner.com/)
%
% Important note:
% This template requires the resume.cls file to be in the same directory as the
% .tex file. The resume.cls file provides the resume style used for structuring the
% document.
%
%%%%%%%%%%%%%%%%%%%%%%%%%%%%%%%%%%%%%%%%%

%----------------------------------------------------------------------------------------
%	PACKAGES AND OTHER DOCUMENT CONFIGURATIONS
%----------------------------------------------------------------------------------------


\documentclass{resume} % Use the custom resume.cls style

\usepackage[T2A]{fontenc}
\usepackage[utf8]{inputenc}
\usepackage[english,russian]{babel}
\usepackage{xcolor}

\usepackage{hyperref}
% Цвета для гиперссылок
\definecolor{linkcolor}{HTML}{799B03} % цвет ссылок
\definecolor{urlcolor}{HTML}{799B03} % цвет гиперссылок
\usepackage[left=0.75in,top=0.6in,right=0.75in,bottom=0.6in]{geometry} % Document margins
\newcommand{\tab}[1]{\hspace{.2667\textwidth}\rlap{#1}}
\newcommand{\itab}[1]{\hspace{0em}\rlap{#1}}
\name{Maxim Zubkov} % Your name
\address{Moscow, Russia} % Your address
%\address{123 Pleasant Lane \\ City, State 12345} % Your secondary addess (optional)
\address{zubkov.md@phystech.edu, $\textcolor{blue}{\href{https://github.com/maximzubkov}{\textnormal{https://github.com/maximzubkov}}}$, $+7(967)$-$120$-$61$-$
12$} % Your phone number and email

\begin{document}

%----------------------------------------------------------------------------------------
%	EDUCATION SECTION
%----------------------------------------------------------------------------------------

\begin{rSection}{Education}

{\bf Moscow Institute of Physics and Technology} \hfill {\em 2017 - 2021} 
\\ Bachelor \hfill { Overall GPA: 8.0/10}
\\ Department of Control and Applied Mathematics  
%Minor in Linguistics \smallskip \\
%Member of Eta Kappa Nu \\
%Member of Upsilon Pi Epsilon \\


\end{rSection}
%----------------------------------------------------------------------------------------
%	TECHNICAL STRENGTHS SECTION
%----------------------------------------------------------------------------------------

\begin{rSection}{Technical Knowledge}

\begin{tabular}{ @{} >{\bfseries}l @{\hspace{6ex}} l }
Programming Languages &  C/C++ (4/5), Python (4/5),  SQL (4/5) \\
Frameworks & Numpy (3/5), Matplotlib (3/5), Pandas (3/5), SciPy (3/5), Requests (3/5) \\
					&  BeautifulSoup (2/5), Asyncio (3/5), STL (4/5), PyQt (1/5)\\
Software \& Tools & LaTeX, Git, Bash, Linux, MS Office, Jupiter, Postgres \\
Soft Skills & Advanced English, Presentation
\end{tabular}

\end{rSection}

%----------------------------------------------------------------------------------------
%	WORK EXPERIENCE SECTION
%----------------------------------------------------------------------------------------

% \begin{rSection}{Working experience}

% \begin{rSubsection}{Systema}{June 2018 - August 2018}{https://sistemavl.ru}{}
% \item Created a website on jQuery with dynamic elements , that are based on Web parsing (distance and price counters). 
% \item Built database architecture
% \end{rSubsection}


%------------------------------------------------

% \end{rSection}

\begin{rSection}{Projects}
\begin{rSubsection}{$\textcolor{blue}{\href{https://github.com/maximzubkov/Parallel}{\textnormal{Shell-like extensions}}}$}{September 2018 - October 2018}{}{}
\item ls, cp command implemention
\item Piping the result of one process to another
\item Posix and sys5 semaphoes
\end{rSubsection}

\begin{rSubsection}{$\textcolor{blue}{\href{https://github.com/maximzubkov/Python_Project}{\textnormal{Users behavior analizer}}}$}{February 2019 - Now}{}{}
\item The main idea is to analize the users behaviour according to his transfers between web-pages
\item To calulate the probobilito of current user to transfer from one web-page to another we used Markov Chains with Hidden Sates
\item Also we developed Google Chrome extension and launched a server that provide multi-user mode
\end{rSubsection}

\begin{rSubsection}{$\textcolor{blue}{\href{https://github.com/maximzubkov/Clique_Find}{\textnormal{Clique Problem}}}$}{March 2019}{}{}
\item Meet In The Middle algorithm
\item Branches and Bounds algorithm
\end{rSubsection}

\begin{rSubsection}{$\textcolor{blue}{\href{https://github.com/maximzubkov/Fourier}{\textnormal{Fast Fourier Transform}}}$}{April 2019}{}{}
\item Implemention of Polynomial class, multiplication, exponentiation using Fast Fourier Transform
\end{rSubsection}

\begin{rSubsection}{$\textcolor{blue}{\href{https://github.com/maximzubkov/MCTS_TicTacToe}{\textnormal{Tic-Tac-Toe}}}$}{April 2019}{}{}
\item Implemention of Tic-Tac-Toe AI using Monte-Carlo Tree Search
\end{rSubsection}

\begin{rSubsection}{$\textcolor{blue}{\href{https://github.com/maximzubkov/Planarity}{\textnormal{Graph Planarity}}}$}{May 2019 - Now}{}{}
\item Check the graph palanrity using Gamma Algorithm
\item Planned for the future to add the graph drawing
\end{rSubsection}

\end{rSection}

%----------------------------------------------------------------------------------------
\begin{rSection}{Relevant Courses}
\itab{\textbf{Mathematical Courses}} \tab{}  \tab{\textbf{Programming and CS Courses}} 
\\ \itab{Calculus } \tab{}  \tab{Introduction to Machine Learning ($\textcolor{blue}{\href{https://www.coursera.org/account/accomplishments/verify/N93LVQYCNYWT}{\textnormal{Coursera}}}$)}
\\ \itab{Linear Algebra} \tab{}  \tab{Operating Systems} 
\\ \itab{Differential Equations} \tab{} \tab{Object-Oriented Programming in C++}
\\ \itab{Combinatorics} \tab{}  \tab{Relational Database Architecture} 
\\ \itab{Graph Theory} \tab{} \tab{Formal Languages}
\\ \itab{Abstract Algebra} \tab{} \tab{Algorithms and Data Structures}
\\ \itab{Lebesgue Measure} \tab{} \tab{Advanced Algorithms and Computation Models}
\\ \itab{Probobility Theory} \tab{} \tab{Automata Theory}
\\ \itab{Physics} \tab{} \tab{Python Language ($\textcolor{blue}{\href{https://www.coursera.org/account/accomplishments/verify/TNSCC4D2WYSX}{\textnormal{Coursera}}}$)}
\\ \itab{Analytical Mechanics} \tab{} \tab{Asynchronous Programming}
\\ \itab{Advanced Abstract Algebra and Number Theory}

% \\ \itab{Process Control (ongoing)} \tab{} \tab{Electrodynamics}

\end{rSection}



\end{document}

