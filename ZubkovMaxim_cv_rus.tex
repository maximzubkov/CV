%%%%%%%%%%%%%%%%%%%%%%%%%%%%%%%%%%%%%%%%%
% Medium Length Professional CV
% LaTeX Template
% Version 2.0 (8/5/13)
%
% This template has been downloaded from:
% http://www.LaTeXTemplates.com
%
% Original author:
% Trey Hunner (http://www.treyhunner.com/)
%
% Important note:
% This template requires the resume.cls file to be in the same directory as the
% .tex file. The resume.cls file provides the resume style used for structuring the
% document.
%
%%%%%%%%%%%%%%%%%%%%%%%%%%%%%%%%%%%%%%%%%

%----------------------------------------------------------------------------------------
%	PACKAGES AND OTHER DOCUMENT CONFIGURATIONS
%----------------------------------------------------------------------------------------


\documentclass{resume} % Use the custom resume.cls style
\usepackage[T2A]{fontenc}
\usepackage[utf8]{inputenc}
\usepackage[english,russian]{babel}
\usepackage{xcolor}
\usepackage{hyperref}
% Цвета для гиперссылок
\definecolor{linkcolor}{HTML}{799B03} % цвет ссылок
\definecolor{urlcolor}{HTML}{799B03} % цвет гиперссылок
\usepackage[left=0.75in,top=0.6in,right=0.75in,bottom=0.6in]{geometry} % Document margins
\newcommand{\tab}[1]{\hspace{.2667\textwidth}\rlap{#1}}
\newcommand{\itab}[1]{\hspace{0em}\rlap{#1}}
\name{Зубков Максим} % Your name
\address{Москва, Россия} % Your address
%\address{123 Pleasant Lane \\ City, State 12345} % Your secondary addess (optional)
\address{zubkov.md@phystech.edu, $\textcolor{blue}{\href{https://github.com/maximzubkov}{\textnormal{https://github.com/maximzubkov}}}$, $+7(967)$-$120$-$61$-$
12$} % Your phone number and email

\begin{document}

%----------------------------------------------------------------------------------------
%	EDUCATION SECTION
%----------------------------------------------------------------------------------------

\begin{rSection}{Образование}

{\bf Московский Физико-Технический Институт} \hfill {\em 2017 - 2021} 
\\ Бакалавр \hfill { Средний балл: 8.0/10}
\\ Факультет Управления и Прикладной Математики, 3 курс, \\Кафедра Анализа Данных
%Minor in Linguistics \smallskip \\
%Member of Eta Kappa Nu \\
%Member of Upsilon Pi Epsilon \\


\end{rSection}
%----------------------------------------------------------------------------------------
%	TECHNICAL STRENGTHS SECTION
%----------------------------------------------------------------------------------------

\begin{rSection}{Знания и Умения}

\begin{tabular}{ @{} >{\bfseries}l @{\hspace{6ex}} l }
Языки программирования &  C/C++ (4/5), Python (4/5),  SQL  (4/5),  \\																				   &JavaSricpt, CSS, HTML (3/5) \\
Frameworks & Numpy (4/5), Matplotlib (3/5), Pandas (5/5), SkLearn (4/5)\\ 
					& NLTK(2/5), Gensim(3/5), FastText(2/5), SpaCy(2/5), PyTorch (3/5)\\ 
					& XGBoost (3/5), Requests (3/5), BeautifulSoup (2/5), Asyncio (3/5)\\
					& Multiprocessing(2/5), STL (3/5), PyQt (1/5), Dash(3/5), ReactJS (3/5)\\
Программы и технологии & LaTeX, Git, Bash, Linux, MS Office, Jupiter, Zeppelin, Postgres \\
Soft Skills & Английский язык уровень Advanced
\end{tabular}

\end{rSection}

%----------------------------------------------------------------------------------------
%	WORK EXPERIENCE SECTION
%----------------------------------------------------------------------------------------

% \begin{rSection}{Working experience}

% \begin{rSubsection}{Systema}{June 2018 - August 2018}{https://sistemavl.ru}{}
% \item Created a website on jQuery with dynamic elements , that are based on Web parsing (distance and price counters). 
% \item Built database architecture
% \end{rSubsection}


%------------------------------------------------

% \end{rSection}

\begin{rSection}{Проекты}

\begin{rSubsection}{$\textnormal{Data Base}$}{Cентябрь 2018 - Декабрь 2018}{}{}
\item Во время прохождения курса "Базы данных" реализовал учебную базу данных, в которой хранится информация о гонщиках Формулы 1, заполнил ее некоторыми данными, добавил к ней триггер, а также написал на языке SQL ее backup и restore
\end{rSubsection}

\begin{rSubsection}{$\textcolor{blue}{\href{https://github.com/maximzubkov/Parallel}{\textnormal{Shell-like extensions}}}$}{Сентябрь 2018 - Октябрь 2018}{}{}
\item Реализация функционала ls, cp из Linux терминала
\item Скрипт, передающий результат выполнения одного процесса в другой
\item Сюжетные задачи на posix и sys5 семафоры, а также message queue
\end{rSubsection}

\begin{rSubsection}{$\textcolor{blue}{\href{https://github.com/maximzubkov/Python_Project}{\textnormal{Users behavior analizer}}}$}{Февраль 2019 - До сих пор}{}{}
\item Проект писался в команде из двух человек, использовались следующие технологии: Марковские цепи, PostgresSQL, Python (библиотеки Numpy, Asyncio, Flask, Pandas, Matplotlib), мой коллега писал на JavaScript соответствующее приложение для Google Chrome, мы вместе поднимали сервер для поддержания многопользовательского режима работы нашего проекта. Мы продолжаем над ним работать и стараемся улучшить имеющуюся модель, так как Google Chrome выдает слишком мало информации об истории пользователя
\end{rSubsection}

\begin{rSubsection}{$\textcolor{blue}{\href{https://github.com/maximzubkov/Clique_Find}{\textnormal{Clique Problem}}}$}{Март 2019}{}{}
\item Решение $\mathcal{NPC}$ задачи о нахождении максимальной клики в графе алгоритмами Meet In The Middle и  Branches and Bounds
\end{rSubsection}

\begin{rSubsection}{$\textcolor{blue}{\href{https://github.com/maximzubkov/Fourier}{\textnormal{Fast Fourier Transform}}}$}{Апрель 2019}{}{}
\item Реализация класса Polynomial, с умножением и возведением в степень при помощи быстрого преобразования Фурье
\end{rSubsection}

\begin{rSubsection}{$\textcolor{blue}{\href{https://github.com/maximzubkov/MCTS_TicTacToe}{\textnormal{Tic-Tac-Toe}}}$}{Апрель 2019}{}{}
\item Реализация игры "крестики нолики" с использованием искусственного интеллекта для противника, основанного на Монте-Карло дереве поиска, которое выбирает наиболее выгодный ход
\end{rSubsection}

\begin{rSubsection}{$\textcolor{blue}{\href{https://github.com/maximzubkov/Planarity}{\textnormal{Graph Planarity}}}$}{Май 2019}{}{}
\item Проверка графа на планарность с использованием Гамма алгоритма
\item В будущем планируется реализовать рисование графа 
\end{rSubsection}

\begin{rSubsection}{$\textcolor{blue}{\href{https://github.com/maximzubkov}{\textnormal{DL in NLP}}}$}{Август 2019 - До сих пор}{}{}
\item Курс от ABBYY по основам NLP
\end{rSubsection}

\end{rSection}

%----------------------------------------------------------------------------------------
\begin{rSection}{Пройденные Курсы}
\itab{\textbf{Математические курсы}} \tab{}  \tab{\textbf{Программирование и CS курсы}} 
\\ \itab{Мат. Анализ } \tab{}  \tab{Введение в Машинное Обучение ($\textcolor{blue}{\href{https://www.coursera.org/account/accomplishments/verify/N93LVQYCNYWT}{\textnormal{Coursera}}}$)}
\\ \itab{Линейная Алгебра} \tab{}  \tab{Операционная система Linux} 
\\ \itab{Дифференциальные уравнения} \tab{} \tab{ООП в C++}
\\ \itab{Комбинаторика} \tab{}  \tab{Реляционные базы данных} 
\\ \itab{Теория Графов} \tab{} \tab{Теория Формальных Систем}
\\ \itab{Высшая Алгебра} \tab{} \tab{Основные Алгоритмы}
\\ \itab{Мера Лебега} \tab{} \tab{Алгоритмы и Модели Вычисления}
\\ \itab{Теория Вероятностей ($\textcolor{blue}{\href{https://www.coursera.org/account/accomplishments/certificate/HYBDXC4X8WGF}{\textnormal{Coursera}}}$)} \tab{} \tab{Теория и Реализация Языков Программирования}
\\ \itab{Физика} \tab{} \tab{Язык Python ($\textcolor{blue}{\href{https://www.coursera.org/account/accomplishments/verify/TNSCC4D2WYSX}{\textnormal{Coursera}}}$)}
\\ \itab{Аналитическая Механика} \tab{} \tab{Многопоточное программирование}
\\ \itab{Избранные вопросы Алгебры и Теории Чисел} \tab{} \tab{Проходил курс $\textcolor{blue}{\href{https://dlcourse.ai}{\textnormal{dlcourse.ai}}}$}
\\ \itab{} \tab{} \tab{Проходил курс $\textcolor{blue}{\href{https://github.com/DanAnastasyev/DeepNLP-Course}{\textnormal{DL in NLP от ABBYY}}}$}

% \\ \itab{Process Control (ongoing)} \tab{} \tab{Electrodynamics}

\end{rSection}

\begin{rSection}{Опыт работы / участия в соревнованиях}

\begin{rSubsection}{Тинькофф Банк}{Июль 2019 - Август 2019}{}{}
\item Работал в департаменте рекомендательных систем, в работе использовал как методы классического машинного обучения (деревья бустинга, алгоритмы кластеризации),  так и методы глубокого обучения (Word2Vec, LSTM)
\end{rSubsection}

\begin{rSubsection}{VK Hack}{Сентябрь 2019}{}{}
\item На хакатоне предлагалось реализовать приложение VK Mini App для Пушкинского музея с голсовым помошником, возможностью прослушивать аудиугиды и навигацией по залам. К сожалению наш проект не занял призовых мест 
\end{rSubsection}

\begin{rSubsection}{CET-MIPT Hack}{Сентябрь 2019}{}{}
\item На хакатоне решалась задача по поиску нефти по данному набору каротажей (времяных рядов). 
\end{rSubsection}

\end{rSection}

\end{document}


